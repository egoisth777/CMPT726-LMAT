\documentclass[11pt]{article}
\usepackage{amsfonts,amssymb,amsmath}
\usepackage{bbold}
\setlength{\evensidemargin}{-.05in}
\setlength{\oddsidemargin}{-.05in}
\setlength{\textwidth}{6in}
%\renewcommand{\baselinestretch}{1.3}
\renewcommand{\theequation}{\thesection .\arabic{equation}}
\setlength{\parskip}{10pt plus 2pt minus 1pt}
\setlength{\parindent}{0pt}
\usepackage{tcolorbox}
\usepackage{float}

\def\Cox{\hfill \Box}
\def\deq{\, {\stackrel {def} {=}}}
\def\diseq{\, {\stackrel {{\cal D}} {=}}}
\def\ee{\epsilon}
\def\B{{\cal B}}
\def\E{{\mathbb{E}}}
\def\P{{\mathbb{P}}}
\def\R{{\mathbb{R}}}
\def\Z{{\mathbb{Z}}}
\def\Q{{\mathbb{Q}}}
\def\F{{\cal{F}}}
\def\B{{\cal B}}
\def\m{{\bf m}}
\def\|{\, | \, }
\def\one{{\bf 1}}
\def\Var{{\rm Var}}
\def\ee{\epsilon}
% Define custom block environment ()

\tcbset{
    myblock/.style={
        colback=blue!5!white, % Background color
        colframe=blue!75!black, % Border color
        width=\textwidth, % Width of the block
        boxrule=0.5mm, % Thickness of the frame
        halign=left, % Horizontal alignment
        fonttitle=\bfseries, % Title formatting
    }
}
\begin{document}

\begin{center}
{\Large \bf CIS 5150 Homework 1} 
\\
\vspace{0.5cm}
{Yueyang Li yl919@sfu.ca}\\
\end{center}
\vspace{10ex}

\section{Linear Algebra}
\subsection{p1.a}

\(A^{-1}\)
\[
A = \begin{bmatrix}
    2 & 0   \\
    0 & 0.5
\end{bmatrix}
\]
Therefore, we can get \(A^{-1}\)
\[
A^{-1} = \frac{1}{\text{det}(A)}\begin{bmatrix}
    0.5 & 0 \\
    0 & 2
\end{bmatrix} = \begin{bmatrix}
    0.5 & 0\\
    0 & 2
\end{bmatrix}
\]

\(B^{-1}\)
\[
B = \begin{bmatrix}
    4 & 3   \\
    2 & 1
\end{bmatrix}
\]
Therefore, we can get \(B^{-1}\)
\[
B^{-1} = \frac{1}{\text{det}(B)}\begin{bmatrix}
    1 & -3 \\
    -2 & 4
\end{bmatrix} = - \frac{1}{2} \cdot \begin{bmatrix}
    1 & -3 \\
    -2 & 4
\end{bmatrix} = \begin{bmatrix}
    -0.5 & 1.5 \\
    1 & -2
\end{bmatrix}
\]



\(C^{-1}\)
\[
C = \begin{bmatrix}
    1 & -2   \\
    -2 & 1
\end{bmatrix}
\]
Therefore, we can get \(C^{-1}\)
\[
C^{-1} = \frac{1}{\text{det}(C)}\begin{bmatrix}
    1 & 2 \\
    2 & 4
\end{bmatrix} = - \frac{1}{3} \cdot \begin{bmatrix}
    1 & 2 \\
    2 & 1
\end{bmatrix} = \begin{bmatrix}
    -\frac{1}{3} & - \frac{2}{3} \\
    - \frac{2}{3} & - \frac{1}{3}
\end{bmatrix}
\]

\subsection{p1.b}

\[
B \cdot C = \begin{bmatrix}
    4 & 3 \\
    2 & 1
\end{bmatrix}
\cdot
\begin{bmatrix}
    1 & -2 \\
    -2 & 1
\end{bmatrix}
= \begin{bmatrix}
    4 \times 1 - 3 \times 2 & 4 \times (-2) + 1 \times 3 \\    
    2 \times 1 - 1 \times 2 & -2 \times 2 + 1 \times 1 
\end{bmatrix}
 = \begin{bmatrix}
   -2 & -5 \\ 
   0 & -3
 \end{bmatrix}
\]

\[
C \cdot B = 
\begin{bmatrix}
    1 & -2 \\
    -2 & 1
\end{bmatrix} \cdot 
\begin{bmatrix}
    4 & 3 \\    
    2 & 1
\end{bmatrix} = 
\begin{bmatrix}
1 \times 4 - 2 \times 2 & 1 \times 3 - 2 \times 1 \\
-2 \times 4 + 1 \times 2 & -2 \times 3 + 1 \times 1  

\end{bmatrix} = 
\begin{bmatrix}
   0 & 1 \\ 
   -6 & -5
\end{bmatrix}
\]



\subsection{p1.c Find Eigenvalues and Eigenvectors}

To find Eigenvalues and Eigenvectors of \(C\), need to compute the following property: 

\[
C \vec{v} = \lambda I \vec{v}
\]


Namely, we are trying to solve:
\begin{equation}
\begin{split}
(C - \lambda I) \vec{v} &= \vec{0} \\ 
D = \begin{bmatrix}
    1 - \lambda & -2 \\
    -2 & 1 - \lambda
\end{bmatrix} \vec{v} &= \vec{0}
\end{split}
\end{equation}


Therefore, \(det(D) = 0 \rightarrow (1-\lambda)^{2} - 4 = 0 \)  \\
Therefore \(\lambda  = -1 \) or \( \lambda  = 3 \)

Then we compute eigenvectors: 
\[
\begin{bmatrix}
    2 & -2 \\
    -2 & 2
\end{bmatrix} \cdot \vec{v} = \vec{0}
\]
\textbf{if \(\lambda = -1\)} \(\vec{v} = \{(x,y) | \text{such that } x = y \}\)
\\
\textbf{if \(\lambda = 3\)} \(\vec{v} = \{(x,y) | \text{such that } x = -y \}\)

\newpage

\section{Calculus}
\subsection{Compute Gradient}

We can define a function

\[
    f(x) = \begin{bmatrix} x_1 & x_2 \end{bmatrix} \begin{bmatrix} 4 & 2 \\ 2 & 0\end{bmatrix} \cdot \begin{bmatrix} x_1 \\ x_2\end{bmatrix} 
    = \begin{bmatrix}
        4 x_{1} + 2 x_{2} & 2 x_{1}
    \end{bmatrix} \cdot \begin{bmatrix}
        x_1 \\
        x_2
    \end{bmatrix}\\
    = 4x_{1}^{2} + 2 x_1 x_2 + 2x_1 x_2 \\
    = 4x_{1}^{2} + 4x_1 x_2       
\]

Therefore
\[
    \frac{\partial{f}}{\partial{x_1}} = 8x_1 + 4x_2 \\
\]
\[
    \frac{\partial{f}}{\partial{x_1}}(1,3) =  8 * 1 + 3 * 4 = 20
\]
\[
    \frac{\partial^{2}{f}}{\partial{x_1}\partial{x_2}} = 4\\
\]
\[
    \frac{\partial^{2}{f}}{\partial{x_1}\partial{x_2}}(2,4) = 4
\]

\subsection{Comput Gradient and Hessian}

\textbf{Gradient}
\[
\nabla f(x) = 
\begin{bmatrix}
    \frac{\partial{f}}{\partial{x_1}} \\
    \frac{\partial{f}}{\partial{x_2}}    
\end{bmatrix} = 
\begin{bmatrix}
    8x_1 + 4x_2 \\    
    4x_1
\end{bmatrix}
\]

\textbf{Hessian}
\[
H_{f}x = 
\left[\begin{array}{cc}
\frac{\partial^2 f}{\partial x_1^2} & \frac{\partial^2 f}{\partial x_1 \partial x_2} \\
\frac{\partial^2 f}{\partial x_2 \partial x_1} & \frac{\partial^2 f}{\partial x_2^2}
\end{array}\right]
\] 
Let's compute the components:\\
$\frac{\partial^2 f}{\partial x_1^2}=\frac{\partial}{\partial x_1}\left(8 x_1+4 x_2\right)=8$ \\
$\frac{\partial^2 f}{\partial x_1 \partial x_2}=\frac{\partial}{\partial x_1}\left(4 x_1\right)=4$ \\
$\frac{\partial^2 f}{\partial x_2 \partial x_1}=\frac{\partial}{\partial x_2}\left(8 x_1+4 x_2\right)=4$ \\
$\frac{\partial^2 f}{\partial x_2^2}=\frac{\partial}{\partial x_2}\left(4 x_1\right)=0$ \\

So, the Hessian matrix is:
\[
H_f(x)=\left[\begin{array}{ll}
8 & 4 \\
4 & 0
\end{array}\right]
\]

\section{Probability}

\subsection{PMF and CDF calculation}

\begin{itemize}
    \item pmf \\
   \( P(X = 1) = \frac{1}{6}\)\\
   \( P(X = 2) = \frac{1}{6}\)\\
   \( P(X = 3) = \frac{1}{6}\)\\
   \( P(X = 4) = \frac{1}{6}\)\\
   \( P(X = 5) = \frac{1}{6}\)\\
   \( P(X = 6) = \frac{1}{6}\)

   \item cdf
   \[
   F_{x}(k) = P(x \leq k) = 
   \begin{cases}
    0   \quad \text{for} k < 1\\
    1/6 \quad \text{for} 1 \leq k < 2 \\
    1/3 \quad \text{for} 2 \leq k < 3 \\ 
    3/6 \quad \text{for} 3 \leq k < 4 \\
    4/6 \quad \text{for} 4 \leq k < 5 \\
    5/6 \quad \text{for} 5 \leq k < 6 \\
    1   \quad \text{for} k \geq 6 \\
   \end{cases}
   \]
 
\end{itemize}






\subsection{3.2 What is \(P(X = 1 | X \text{is Odd})\)}
\[
P(X \text{ is Odd}) = 1/3 
\]

\[
P(X = 1 \cap \text{X is Odd}) = P(X = 1) = 1/6
\]

Therefore:
\[
P(X = 1 | X \text{ is Odd}) = \frac{1/6}{1/3} = 1/3
\]


\subsection{}


Since \(\{ X_1, X_2, \dots, X_n\}\) are \textbf{independent random variables}

\textbf{E[X]}
\[
E[X]=\sum_{k=1}^6 k \cdot P(X=k)=(1+2+3+4+5+6) \cdot \frac{1}{6}=\frac{21}{6}=3.5
\]


We know that \(S = X_1 + X_2 + \dots + X_n\), where \(X_i\) is independent random variable 

\textbf{E[S]}:\\
\[
E[S] = E[\sum_{i = 1}^{n}X_{i}] = \sum_{i = 1}^{n}E[X_{i}]= n * E[X] = \frac{7}{2}  \cdot n
\]



\textbf{Var(X)}:\\


The variance is $\operatorname{Var}(X)=E\left[X^2\right]-(E[X])^2$.

$$
\begin{aligned}
& E\left[X^2\right]=\sum_{k=1}^6 k^2 \cdot P(X=k)=\left(1^2+2^2+3^2+4^2+5^2+6^2\right) \cdot \frac{1}{6}=(1+4+9+ \\
& 16+25+36) \cdot \frac{1}{6}=\frac{91}{6} \\
& \operatorname{Var}(X)=\frac{91}{6}-(\frac{21}{6})^2=\frac{91}{6}-\left(\frac{7}{2}\right)^2=\frac{91}{6}-\frac{49}{4}=\frac{182}{12}-\frac{147}{12}=\frac{35}{12}
\end{aligned}
$$



\textbf{Variance of \(S\)} : \\
Because the die rolls are independent, the variance of the sum is the sum of the variances.

$$
\begin{aligned}
& \operatorname{Var}(S)=\operatorname{Var}\left(X_1+\cdots+X_n\right)=\operatorname{Var}\left(X_1\right)+\cdots+\operatorname{Var}\left(X_n\right)=n \cdot \operatorname{Var}(X) \\
& \operatorname{Var}(S)=n \cdot \frac{35}{12}
\end{aligned}
$$


\end{document}

